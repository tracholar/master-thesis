\chapter{绪论}
\section{研究背景简介}

\subsection{量子信息技术简介}
在人类对世界的认知过程当中,认知的尺度从宏观到微观,从分子尺度到原子、电子尺度。
物理实体所遵循的物理定律也从牛顿力学到量子力学。
另一方面,在近代科学的发展过程当中,通信科学和信息技术是的发展对人类影响最大的学科分支之一。
经典的通信一般采用电磁波为载体,在有线或无线信道中传输。
描述经典通信中的物理规律是麦克斯韦方程组,
它们是由麦克斯韦从前人的实验中总结出来的电磁学规律。

\subsection{量子接收机研究意义}
量子信息技术和光通信中,都会存在一个基本问题,那就是如何有效地区分相干态。
在量子力学中,人们用相干态来描述物理上的一束相干光脉冲。

在经典光通信系统中,由于受到散粒噪声的影响,
通信系统接收机性能存在着经典极限——标准量子极限(standard quantum limit, SQL)。
这个极限可以从经典假设检验出发,分析经典光接收机的性能得到。
近年来,随着量子技术的发展,利用光的量子特性设计新型的光接收机方案逐渐引起学术界的关注。
这是因为,采用量子探测和测量的技术可以获取之前使用经典探测所不能获得的信息。
采用量子资源进行探测最先始于量子通信\cite{gisin2007quantum}
和量子计算\cite{ladd2010quantum}。
而采用量子检测的技术对非正交相干态的区分,
可以让系统的性能突破标准量子极限,
从而获得经典检测方案所不能达到的更低的误码率\cite{helstrom1976quantum}和更高的信道容量\cite{hausladen1996classical}。

研究量子检测的物理实现最终归结于量子接收机的设计,
研究量子接收机的意义十分重要。
量子接收机致力于对相干态进行区分,相干态是一种非常理想的信息载体,
因为相干态能够在有损介质中传输仍然保持相干态,
并且能够有效地恢复发射状态。
另一方面,利用相干态通信已经很多年了,取得了丰硕的成果,
具有很高的谱效率,目前基于相干态通信可以达到Tbps的量级\cite{jinno2007networks}。
但是随着大数据时代的到来,人们对通信速率和带宽的需求不断增加,
工艺和技术的不断进步使得基于经典理论实现的零差和外差接收机已
不断的逼近SQL\cite{tsukamoto2006unrepeated},
因此研究基于量子检测与估计理论的量子接收机变得越来越重要。
此外,在深空通信领域,由于深空通信对通信系统体积和能量效率的要求,
使得采用激光通信成为替代无线电进行深空通信的手段\cite{hemmati2006deep}。
量子接收机能在很大程度上提高系统的能量效率,
因为在使用相同的功率和给定的误码率的情况下,
采用量子接收机工作距离比传统的光接收方案高得多\cite{helstrom1976quantum}。



\subsection{量子接收机的国内外研究现状与趋势}
对于量子接收机的研究要追溯到上个世纪六十年代,
C. W. Helstrom, A. S. Holevo, H.P. Yuen等人首先从理论上分析了
在量子力学的框架下,能够实现的接收机性能极限
——Helstrom极限\cite{helstrom1976quantum,helstrom1967detection,
yuen1975optimum,eldar2003designing}。
结果表明,标准量子极限是可以突破的,
并且理论上存在比标准量子极限好得多的接收方案。

但是直到1973年,R. S. Kennedy才从理论上提出
第一种利用现有的光学元器件可以实现的接收机方案——Kennedy接收机\cite{kennedy1973near}。
Kennedy接收机将BPSK信号通过一个本振光信号进行位移操作,
转化为OOK信号,然后利用单光子探测器进行光强探测。
这种方案在光子数比较大的区域可以突破标准量子极限,
但是在光子数较小的情况下还不如零差接收的性能。

后来,Kennedy的学生Dolinar在这个基础上增加了反馈控制策略\cite{dolinar1973optimum}。
这个方案在任何区域都可以突破标准量子极限,并且更加令人振奋的是,
这个方案对BPSK和OOK来说,理论上可以达到Helstrom极限。
实验上的论证直到2007年由Cook等人论证\cite{cook2007optical},
但是由于实验条件的限制,实验结果没有达到Helstrom极限,
但是在一个区域内可以突破标准量子极限。

1993年,Bondurant受到Dolinar的启发,
设计出了第一个QPSK信号的量子接收机\cite{bondurant1993near}。
该方案采用位移和反馈操作,顺序归零某个符号,
同时进行光子计数,根据光子计数结果控制需要归零的符号。
他的方案可以在光子数较大的区域突破标准量子极限,
但是离Helstrom极限还有一段距离。
到了2014年,Müller 改进了Bondurant的位移策略,
这个方案没有采用精确归零,而是采用一种最优归零策略。
在光子数较大的时候,最优归零近似精确归零,但是在光子数较小的时候,
最优归零更接近外差接收,位移量远大于信号强度,
同时采用最大后验概率的归零顺序。
在这个策略下,改进后的Bondurant接收机可以在任意光子数下
突破标准量子极限\cite{muller2014qpsk}。
接着,Müller将这个方案推广到任意PSK信号,得到了类似的结果\cite{muller2014m}。

由于Bondurant接收机和Dolinar接收机一样需要实时反馈控制,
这对工程实践提出了很大挑战。
2011年,马里兰大学的Becerra从Bondurant接收机得到启发,
简化了实时反馈控制,采用分区反馈控制,在每一个分区内,
位移量是恒定的,这样就减少了带宽的使用。
同时采用最大后验概率归零顺序,提升接收机性能\cite{becerra2011m}。
在2013年,该团队采用反馈方式实现了上述实验方案\cite{becerra2013experimental},
证实了该方案实验的可行性。
与此同时,日本研究组的Sasaki也从理论上提出
了M阶PSK位移接收机方案\cite{izumi2012displacement},
并对3PSK和QPSK进行数值仿真,分析了非理想因素对位移接收机的影响。
2013年,该团队又提出了一种增加了压缩操作的接收机,
同时采用光子数可分辨的探测器(PNRD)减少暗计数
对接收机的影响\cite{izumi2013quantum}。
在此基础上,李科等人分析了PNRD对模式适配的影响,
发现PNRD能够有效的克服模式适配问题,提高接收机的鲁棒性\cite{li2013suppressing},
该结果由Becerra在2014年进行验证\cite{becerra2015photon}。

与上述思路不同的是,德国研究者Müller从混合接收的思路出发,
针对QPSK信号,提出了基于零差检测和
最优Kennedy接收机的方案\cite{muller2012quadrature}。
该方案首先利用一个零差接收机,将信号区分到相空间中的上半空间
和下半空间,接着利用零差接收机的结果,进行位移操作,
将问题变成BPSK信号区分问题。
对于这个子问题,采用最优Kennedy接收方案。
最优Kennedy方案是在Kennedy接收机中对位移量进行优化得到的。
沿着这个思路,李科在他的博士论文中提出了基于零差接收
和分区检测接收机的16QAM量子接收机,通过理论分析和数值仿真,
证实这个方案可以突破标准量子极限\cite{李科2014}。

对于PPM信号,Dolinar在1982年第一次提出PPM信号的条件脉冲归零(CPN)
量子接收机\cite{dolinar1982near}。对于第一个时隙,
通过一个位移操作将脉冲归零。如果第一个时隙没有探测到光子计数,
接下来M-1个时隙就直接检测。如果第一个时隙有光子计数,
那么说明第一个时隙应该是0,接下来把剩下的M-1个时隙当做M-1阶PPM信号
进行探测即可,整个决策过程可以用一颗二叉决策树来描述。
2011年,Chen等人从实验上论证了CPN接收机的可行性,
并且对CPN采用非精确归零的方法加以改进\cite{chen2012optical}。

在量子信道传输经典信息是量子信息领域一个基本问题,
那么通过给定的量子资源,所能传递的经典信息的上限是多少
是一个很重要的问题。
Paul Hausladen、Holevo等人首先从理论上论证不论所用的量子态是纯态
还是一般的混态,通过量子信道传递的经典信息量
不超过Holevo信息\cite{hausladen1996classical,holevo1996capacity}。
如何设计出合适的调制方案和检测方案一直困扰着研究者,
Paul Hausladen提出一种“超密编码”方案\cite{hausladen1996classical},
对于纯态信号可以达到Holevo容量,
极化编码是另外一种实现Holevo容量的方案\cite{wilde2013polar,guha2012polar}。
由于单符号信道通常都无法达到Holevo容量,所以通常需要对多个符号进行联合编码和译码。
2012年Lloyd提出序贯译码方案,每一步通过一个投影测量,
然后利用序贯译码策略进行译码,
最终性能能够逼近Holevo容量\cite{giovannetti2012achieving}。
但是如何物理上实现这些测量方案,研究得不是很多。

国际上对量子接收机的物理实现上的研究方向主要集中在一下几点:
对经典的反馈和归零接收机进行改进,
通过策略上的优化如动态规划\cite{dalla2014adaptive},
因为接收机的位移操作可以看做一个控制变量,
反馈控制策略实际上就是根据输出对控制量进行调整,
整个过程可以看做一个最优控制问题进行优化;
另一方面,通过增加其他光学元件如相敏放大\cite{guha2011approaching},
在反馈控制中增加一个压缩操作,通过优化压缩操作的参数,
进一步改善接收机的性能。




\section{论文主要内容}
\subsection{QAM信号量子接收机理论研究主要内容}
到目前为止,对二元信号区分如二元相移键控(BPSK)和开关监控(OOK)的
接收机研究比较成熟\cite{helstrom1976quantum,kennedy1973near,dolinar1973optimum,cook2007optical},
Dolinar接收机能够突破SQL,逼近Helstrom极限\cite{dolinar1973optimum}。
对多元信号区分的量子接收机研究一直在发展中,
至今为止还没有找到对多个符号,
能够达到Helstrom极限的物理可实现的量子接收机,
因此研究多符号量子接收机的意义十分重要。
到目前为止,研究人员提出了针对相移键控(PSK)的位移接收\cite{bondurant1993near,becerra2011m,izumi2012displacement}
和混合接收方案\cite{muller2012quadrature}。
但是针对QAM信号的研究还比较少,因此在本文中,
我将应用于PSK信号的Bondurant接收、分区自适应接收以及混合接收方案
推广到QAM信号,研究在这种信号的配置下,
这三种接收机是否能够突破标准量子极限。


首先,我们研究最早被应用到QPSK信号接收的Bondurant接收机方案,
这种接收机按照给定顺序,通过控制本振场,依次将每个符号归零,
利用光子计数探测是否接收到光子,并实时反馈到本振场。
通过对接收机进行数学建模,进行理论推导,求解出最终的平均错误概率。
结果分别于经典外差接收方案(标准量子极限)
和理论的Helstrom极限进行对比,分析这种接收机的理论性能。

然后,通过对Bondurant接收机的改进,将实时反馈控制改为有限的时间分区,
同时将顺序归零改为按照最大后验概率的自适应归零策略。
在这种设置情况下,理论推导过于复杂,
所以我们采用了蒙特卡洛仿真的方法,得到这种接收机的性能曲线。

另一方面,混合接收机也被应用于QPSK信号的接收,
它包括一个零差接收和一个Kennedy接收机。
这种接收机能够在突破标准量子极限的同时,简化接收策略。
我将这种混合方案推广到QAM信号,研究混合方案对QAM信号接收的性能。

通过对比前面三种接收方案,得到一些可以用于设计量子接收机的一些结论,
例如分析反馈带宽、自适应策略、最优位移以及混合接收
对最终性能的影响的大小。

\subsection{二元编码信号的联合译码量子接收机理论研究主要内容}

分析完多元高阶调制信号的量子接收机之后,
我将目光放到编码过后的信号上来。
之前不论对OOK、PSK还是QAM,都只针对单个符号的接收。
在量子信息理论中,量子信道的Holevo容量是指在给定的符号集合下,
信道所能传递的最大交互信息量。
这个容量通常采用单个符号是无法达到的,必须通过恰当编码,
同时在接收端对无限长的码字做联合接收,
所以研究对多个符号的联合接收十分重要\cite{holevo1996capacity,hausladen1996classical}。
在本文中,我选取二元调制OOK和BPSK编码后的信号,
利用条件归零(CPN)接收机进行接收。
条件归零接收机最初被用来检测脉冲位置调制(PPM)信号\cite{dolinar1982near},
而PPM信号可以看做OOK信号通过一种特殊的编码得到的信号。
因此,很容易想到可以将这种接收方案应用到一般的OOK信号编码后信号
的接收。更一般的,
这种接收机是否能应用到更一般的二元编码信号的联合译码接收
是一个非常值得研究的问题。

首先我们对条件归零接收机建立数学模型,在每一个时间隙内,
采用Kennedy接收机的模型建模。
然后建立一个代价函数,通过优化这个代价函数,
对不同时间隙的控制参数进行优化。
由于这个代价函数可以分解为两个子问题求解,
所以可以通过动态规划进行优化。
最后,通过数值仿真可以得到这种优化后的接收机在不同信号下的接收性能。

\subsection{量子接收机实验平台的搭建主要内容}
国际上对量子接收机实验的研究发展比理论稍晚,经过近几十年的发展,
已经取得了很多里程碑式的进展。
比如二元调制的Kennedy接收机\cite{lau2006binary}、
Dolinar接收机\cite{lau2006binary,cook2007optical},
QPSK的混合接收机以及自适应反馈接收机等都有了实验的验证\cite{muller2012quadrature,becerra2013experimental}。
但是目前国内进行的实验研究尚未见诸报道,
我们实验室在调研和理论研究的基础上,开始准备量子接收机的实验研究。
在研究初期,我们需要搭建量子接收机的实验研究平台。
作为毕业论文的一部分,我参与到实验平台的搭建中。
由于实验周期比较长,所以论文只会选取一个最简单的接收方案,
进行初步的实验。

在实验平台的搭建初期,我们以C. W. Lau 2006年的文章和F. E. Becerra 2013年与2015年的
两篇实验文章为参考\cite{lau2006binary,becerra2013experimental,becerra2015photon},搭建自己的实验平台。
主要工作包括器件的调研、选型,以及器件参数的确定和相关理论仿真验证。 


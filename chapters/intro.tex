\chapter{绪论}
\section{研究背景简介}

\subsection{量子接收机研究意义}
在人类对世界的认知过程当中,认知的尺度从宏观到微观,从分子尺度到原子、电子尺度。
物理实体所遵循的物理定律也从牛顿力学到量子力学。
另一方面,在近代科学的发展过程当中,通信科学和信息技术已发展为对人类影响最大的学科分支之一。
经典的通信系统一般采用电磁波为载体,在有线或无线信道中传输信息。
描述经典通信中的物理规律是麦克斯韦方程组,
它们是由麦克斯韦从前人的实验中总结出来的电磁学规律。
受限于对光的电磁学理论,经典系统中所采用的接收方案
被限定在直接检测、零差检测和外差检测等有限的几种检测方案之中。
在量子信息技术和光通信中,都会存在一个基本问题,如何有效地区分相干光脉冲。
在量子力学中,相干光脉冲常被数学描述为相干态\cite{glauber1963coherent}。
在经典光通信系统中,由于受到散粒噪声的影响,
通信系统接收机性能存在着经典极限——标准量子极限(standard quantum limit, SQL)。
这个极限可以从经典假设检验出发,分析经典光接收机的性能得到\cite{helstrom1976quantum}。
近年来,随着量子信息技术的发展,利用光的量子特性设计新型的光接收机方案逐渐引起学术界的关注,
人们对探测方案的认识也被扩展到一般的正定算子值测量(POVM),
人们意识到采用量子探测和测量的技术可以获取之前使用经典探测所不能获得的信息。
二十世纪六十年代,Helstrom等人在量子力学和检测与估计理论发展出一套量子检测预估计理论,
并给出了在量子力学框架下最优检测的数学形式,发现量子最优检测能够非常有效的降低接收机的平均错误概率,
突破标准量子极限,提升系统的性能\cite{helstrom1976quantum,helstrom1967detection,yuen1970optimal, yuen1975optimum}。
采用量子最优检测的系统可以获得经典检测方案所不能达到的更低的误码率\cite{helstrom1976quantum}和更高的信道容量\cite{hausladen1996classical}。
研究人员逐渐发现,
这种利用量子资源进行探测的方法不但可以应用于提升经典光通信系统的性能\cite{helstrom1976quantum},
还可以应用于量子通信\cite{gisin2007quantum}和量子计算\cite{ladd2010quantum}。



另一方面,从人们对通信速度的需求上来看,研究量子接收机的意义也十分重要。
量子接收机致力于对相干态进行区分,相干态是一种非常理想的信息载体,
因为相干态能够在有损介质中传输仍然保持相干态,只是存在幅度的衰减和相位移动,
所以接收端仍然可以有效地恢复发送的信息。
利用相干态通信已经很多年了,取得了丰硕的成果,
具有很高的谱效率,目前基于相干态通信可以达到Tbps的量级\cite{jinno2007networks}。
但是随着大数据时代的到来,人们对通信速率和带宽的需求不断增加,
工艺和技术的不断进步使得基于经典理论实现的零差和外差接收机已
不断的逼近SQL\cite{tsukamoto2006unrepeated},
因此研究基于量子检测与估计理论的量子接收机提升通信系统性能变得越来越重要。
此外,在深空通信领域,由于深空通信对通信系统体积和能量效率要求很高,
相比于无线电,激光因为具有很小的发散角,所以具有比无线电低得多的衰减,
这使得采用激光通信成为替代无线电进行深空通信的通信手段\cite{hemmati2006deep}。
因为在使用相同的功率和给定的误码率的情况下,
采用量子接收机工作距离比传统的光接收方案高得多\cite{helstrom1976quantum},
这使得量子接收机能在很大程度上提高系统的能量效率,
因而非常适合于深空通信场景。



\subsection{量子接收机的国内外研究现状与趋势}
对于量子接收机的研究要追溯到上个世纪六十年代,
C. W. Helstrom, A. S. Holevo, H.P. Yuen等人首先从理论上分析了
在量子力学的框架下,能够实现的接收机性能极限
——Helstrom极限\cite{helstrom1976quantum,helstrom1967detection,yuen1970optimal,
yuen1975optimum,eldar2003designing}。
结果表明,标准量子极限是可以突破的,
并且理论上存在比标准量子极限好得多的接收方案。
但是直到1973年,R. S. Kennedy才从理论上提出
第一种利用现有的光学元器件可以实现的接收机方案——Kennedy接收机\cite{kennedy1973near}。
Kennedy接收机将BPSK信号通过本振光信号进行位移操作,
转化为OOK信号,然后利用单光子探测器进行光强探测。
理想情况下,这种方案在光子数比较大的区域可以突破标准量子极限,
但是在光子数较小的情况下性能不佳,没能突破标准量子极限。

后来,Kennedy的学生Dolinar在Kenendy接收机的基础上增加了反馈控制策略\cite{dolinar1973optimum}。
理想情况下,这个方案在任何光子数区域都可以突破标准量子极限,并且令人振奋的是,
该方案对于BPSK和OOK信号,可以达到终极量子极限——Helstrom极限。
直到2007年,Cook等人才完成对Dolinar接收机进行实验上的论证\cite{cook2007optical}。
由于当时实验条件的限制,实验结果没有达到Helstrom极限,
但是在一个区域内可以突破标准量子极限。

1993年,Bondurant受到Dolinar的启发,
设计出了第一个QPSK信号的量子接收机\cite{bondurant1993near}。
该方案采用位移和反馈操作,顺序归零4个符号,
同时进行光子计数,根据光子计数结果控制需要归零的符号。
他的方案可以在光子数较大的区域突破标准量子极限,
但是离Helstrom极限还有一段距离。
到了2014年,Müller 改进了Bondurant的位移策略,
这个方案没有采用精确归零,而是采用一种最优归零策略。
在光子数较大的时候,最优归零策略近似为精确归零,但是在光子数较小的时候,
最优归零更接近外差接收,位移量远大于信号强度。
此外,新的策略还同时采用最大后验概率的归零顺序,在每一次光子计数事件的时候,
计算出4个符号的后验概率,选择后验概率最大的符号进行归零。
在这个策略下,改进后的Bondurant接收机可以在任意光子数区域
突破标准量子极限\cite{muller2014qpsk}。
接着,Müller将这个方案推广到任意PSK信号,也得到了类似的结果\cite{muller2014m}。

由于Bondurant接收机和Dolinar接收机一样需要实时反馈控制,
这对工程实践提出了很大挑战。
2011年,马里兰大学的Becerra从Bondurant接收机得到启发,
简化了实时反馈控制,采用分区前馈控制,在每一个分区内,
位移量是恒定的,这样就大大减少了对前馈带宽的要求。
在他的控制策略中,也采用了最大后验概率归零,该策略有效地提升了接收机性能\cite{becerra2011m}。
在2013年,该团队采用反馈方式实现了上述实验方案,
证实了该方案实验的可行性,实验结果表明在实验条件的非理想因数下,
该方案可以在较大的一个区域突破标准量子极限,最大增益达到6dB\cite{becerra2013experimental}。
与此同时,日本研究组的Sasaki也从理论上提出
了M阶PSK位移接收机方案,
并对3PSK和QPSK进行数值仿真,分析了非理想因素对位移接收机的影响\cite{izumi2012displacement}。
2013年,该团队又提出了一种增加了压缩操作的接收机,
同时采用光子数可分辨的探测器(PNRD)减少暗计数
对接收机的影响\cite{izumi2013quantum}。
在此基础上,李科等人分析了PNRD对模式失配的影响,
发现PNRD能够有效的克服模式失配问题,提高接收机的鲁棒性\cite{li2013suppressing},
该结果由Becerra在2014年进行验证\cite{becerra2015photon}。

与上述思路不同的是,德国研究者Müller从混合接收的思路出发,
针对QPSK信号,提出了基于零差检测和
最优Kennedy接收机的混合接收方案\cite{muller2012quadrature}。
该方案首先利用一个零差接收机,将信号区分到相空间中的上半空间
和下半空间,接着利用零差接收机的结果,进行位移操作,
将问题变成BPSK信号区分问题。
对于这个子问题,他采用了最优Kennedy接收方案。
最优Kennedy方案是在Kennedy接收机中对位移量进行优化得到的。
沿着这个思路,李科在他的博士论文中提出了基于零差接收
和分区检测接收机的16QAM量子接收机,通过理论分析和数值仿真,
证实这个方案也可以突破标准量子极限\cite{李科2014}。

对于PPM信号,Dolinar在1982年第一次提出PPM信号的条件脉冲归零(CPN)
量子接收机\cite{dolinar1982near}。对于第一个时隙,
通过一个位移操作将脉冲归零。如果第一个时隙没有探测到光子计数,
接下来M-1个时隙就直接检测。如果第一个时隙有光子计数,
那么说明第一个时隙应该是0,接下来把剩下的M-1个时隙当做M-1阶PPM信号
进行探测即可,整个决策过程可以用一颗二叉决策树来描述。
2011年,Chen等人从实验上论证了CPN接收机的可行性,
并且对CPN采用非精确归零的方法加以改进\cite{chen2012optical}。

在量子信道传输经典信息是量子信息领域一个基本问题,
那么通过给定的量子资源,所能传递的经典信息的上限是多少
是一个很重要的问题。
Paul Hausladen、Holevo等人首先从理论上论证不论所用的量子态是纯态
还是一般的混态,通过量子信道传递的经典信息量
不超过Holevo信息\cite{hausladen1996classical,holevo1996capacity}。
如何设计出合适的调制方案和检测方案一直困扰着研究者,
Paul Hausladen提出一种“超密编码”方案\cite{hausladen1996classical},
对于纯态信号可以达到Holevo容量,
极化编码是另外一种实现Holevo容量的方案\cite{wilde2013polar,guha2012polar}。
由于单符号信道通常都无法达到Holevo容量,所以通常需要对多个符号进行联合检测和译码。
2012年Lloyd提出序贯译码方案,每一步通过一个投影测量,
然后利用序贯译码策略进行译码,
最终信道容量能够逼近Holevo容量\cite{giovannetti2012achieving}。
但是如何物理上实现这些测量方案,研究得不是很多。

国际上对量子接收机的物理实现上的研究方向主要集中在一下几点:
通过策略上的优化如动态规划\cite{dalla2014adaptive},
对经典的反馈和归零接收机进行改进,
因为接收机的位移操作可以看做一个控制变量,
反馈控制策略实际上就是根据输出对控制量进行调整,
整个过程可以看做一个最优控制问题进行优化;
另一方面,通过增加其他光学元器件如相敏放大\cite{guha2011approaching},
在反馈控制中增加一个压缩操作,通过优化压缩操作的参数,
进一步改善接收机的性能。




\section{论文主要内容}
\subsection{QAM信号量子接收机理论研究主要内容}
到目前为止,对二元信号区分如二元相移键控(BPSK)和开关监控(OOK)的
接收机研究,
目前已经有较多的研究工作发表出来\cite{helstrom1976quantum,kennedy1973near,dolinar1973optimum,cook2007optical},
例如Dolinar接收机能够突破SQL,逼近Helstrom极限,从理论上来说已经解决了
二元调制信号的最优量子检测方案的物理实现问题\cite{dolinar1973optimum}。
而至今为止还没有找到对多个符号
也能够达到Helstrom极限的物理可实现的量子接收机,
因此研究多符号量子接收机的意义十分重要。
虽然如此,但是一些能够突破标准量子极限近最优的接收机结构已相继被提出,
到目前为止,研究人员提出了针对相移键控(PSK)的位移接收方案\cite{bondurant1993near,becerra2011m,izumi2012displacement}
和混合接收方案\cite{muller2012quadrature}。
但是针对QAM信号的研究还比较少,因此在本文中,
我将PSK信号的Bondurant接收、分区自适应接收以及混合接收方案
推广到QAM信号,研究在这种信号的配置下,
这三种接收机是否也能够突破标准量子极限。


首先,我们研究最早被应用到QPSK信号的Bondurant接收机,
这种接收机按照给定顺序,通过控制本振场,依次将每个符号归零,
利用光子计数探测是否接收到光子,并实时反馈到本振场。
通过对接收机进行数学建模,进行理论推导,求解出最终的平均错误概率。
结果分别与经典外差接收方案(标准量子极限)
和Helstrom极限进行对比,分析这种接收机的理论性能否突破标准量子极限。

然后,通过对Bondurant接收机的改进,将实时反馈控制改为有限的时间分区,
同时将顺序归零改为按照最大后验概率的自适应归零策略。
在这种设置情况下,理论推导过于复杂,
所以我们采用了蒙特卡洛仿真的方法,得到这种自适应反馈控制接收机的性能曲线。

另一方面,我们也将混合接收机应用于QAM信号的接收,
它包括一个零差接收和一个位移接收机。
这种接收机能够在突破标准量子极限的同时,简化接收策略。
我通过理论分析和数值仿真研究该混合方案对QAM信号接收的理论性能。

通过对比前面三种接收方案,得到一些可以用于设计量子接收机的结论,
例如分析反馈带宽、自适应策略、最优位移以及混合接收
对最终性能的影响的大小。

\subsection{二元编码信号的联合译码量子接收机理论研究主要内容}

分析完多元高阶调制信号的量子接收机之后,
我们将目光放到编码后的信号上来。
之前不论对OOK、PSK还是QAM,都只针对单个符号的接收。
在量子信息理论中,量子信道的Holevo容量是指在给定的符号集合下,
信道所能传递的最大交互信息量。
这个容量通常采用单个符号是无法达到的,必须通过恰当编码,
同时在接收端对无限长的码字做联合接收,
所以研究对多个符号的联合接收十分重要\cite{holevo1996capacity,hausladen1996classical}。
在本文中,我选取二元调制OOK和BPSK编码后的信号,
利用条件归零(CPN)接收机进行接收。
条件归零接收机最初被用来接收脉冲位置调制(PPM)信号\cite{dolinar1982near},
而PPM信号可以看做OOK信号通过一种特殊的编码得到的信号。
因此,很容易想到可以将这种接收方案应用到一般的OOK信号编码后信号
的接收。更一般的,
这种接收机是否能应用到更一般的二元编码信号的联合译码接收
是一个非常值得研究的问题。

本文中,首先我们对条件归零接收机建立数学模型,在每一个时间隙内,
采用Kennedy接收机的模型建模。
然后建立一个代价函数,通过优化这个代价函数,
对不同时间隙的控制参数进行优化。
由于这个代价函数可以分解为两个子问题求解,
所以可以通过动态规划对这些控制参数进行优化。
最后,通过数值仿真可以得到这种优化后的接收机在不同信号下的接收性能。

\subsection{量子接收机实验平台的搭建主要内容}
国际上对量子接收机实验的研究发展比理论稍晚,经过近几十年的发展,
已经取得了很多里程碑式的进展。
比如二元调制的Kennedy接收机\cite{lau2006binary}、
Dolinar接收机\cite{lau2006binary,cook2007optical},
QPSK的混合接收机以及自适应反馈接收机等都有了实验的验证\cite{muller2012quadrature,becerra2013experimental}。
但是目前国内进行的实验研究尚未见诸报道,
我们实验室在调研和理论研究的基础上,开始准备量子接收机的实验研究。
在研究初期,我们需要搭建量子接收机的实验研究平台。
并通过初步的实验验证量子接收机理论。

在本文中,我们以C. W. Lau 2006年的文章和F. E. Becerra 2013年与2015年的
两篇实验文章为参考\cite{lau2006binary,becerra2013experimental,becerra2015photon},搭建自己的实验平台。
主要工作包括器件的调研、选型,以及器件参数的确定和相关理论仿真验证。
最后,在搭建好的实验平台上,对BPSK调制的Kennedy接收机进行初步实验验证,
为日后进一步实验提供经验和参考。
 


\begin{abstract}
在经典光通信系统中,由于受到散粒噪声的影响,
通信系统接收机性能存在着经典极限——标准量子极限(standard quantum limit, SQL)。
近年来,随着量子技术的发展,利用光的量子特性设计新型的光接收机方案逐渐引起学术界的关注。
这是因为,采用量子探测和测量的技术可以获取之前使用经典探测所不能获得的信息,
这种方案可以让系统的性能突破标准量子极限,
从而获得经典检测方案所不能达到的更低的误码率和更高的信道容量。
当前对量子接收机的研究进展还比较缓慢,尚停留在理论研究和实验演示验证阶段,
仍然有许多值得研究的理论课题和试验需要研究。


在本论文中,我们首先回顾一下自上个世纪六十年代以来的研究进展,
然后就其中的几个问题进行深入探究。本文主要分为以下三个部分。


1. QAM信号量子接收机理论研究。
到目前为止,二元调制和相位调制(PSK)方案的量子接收机,
国际上已经有较多的研究人员关注,也取得了很多具有突破意义
的成果。但是对于更高阶调制的QAM信号,研究人员关注的还较少。
本文将目前几种接收机方案推广到QAM信号,对QAM信号的量子接收机
进行系统的研究。这些接收机包括Bondurant接收机、
自适应分区检测接收机和混合接收机等三种接收机方案。
并且对三种接收机方案进行对比,为实际工程上的量子接收机设计
提供参考。

2. 二元调制多符号信号条件归零接收机理论研究。
相比于对单个符号的检测方案,采用联合检测思想的条件归零接收机
更具有研究价值。在本文中,我们将条件归零接收机的方案
应用到多脉冲脉冲位置调制(MPPM)、编码后的OOK调制、
编码后的BPSK调制信号。在此基础上,通过对接收机接收策略的优化,
我们将接收机的性能降低到经典检测方案以下。

3. 量子接收机实验平台的搭建。






\keywords{量子接收机\zhspace 标准量子极限\zhspace QAM调制\zhspace 
量子编码\zhspace 光通信}

\end{abstract}


\begin{enabstract}
In the classical optical communication systems, due to the influence of shot noise,
the performances of the communication system receiver are limited by the standard quantum limit (SQL).
In recent years, with the development of quantum technology, 
the optical receivers using quantum properties of light are concerned by academics.
Because the quantum detection and measurement of the use of information technology can not be obtained by classical probe before you can get,
This scheme allows the performance of the system to break the standard quantum limit,
Thereby obtaining classic detection scheme can not achieve lower bit error rate and higher channel capacity.
Current Advances in quantum receiver is still relatively slow, still remain in the theoretical study and experimental demonstration stage,
There are still many issues worthy of study theoretical and experimental research needs.


In this paper, we first review the research progress made since the 1960's,
Then a few of these issues in-depth inquiry. This paper is divided into three parts.


1. The theory of quantum receiver QAM signal.
So far, quantum two yuan and phase modulation (PSK) scheme receiver,
International researchers have been more concerned, but also made a lot of ground-breaking
Results. But for higher order QAM modulated signals, the researchers also less concern.
This article will present several programs to promote the QAM signal receiver for QAM signals quantum receiver
Research system. The receiver includes a receiver Bondurant,
Three kinds of adaptive partitioning scheme receiver detection receiver and hybrid receivers.
And the three receiver scheme compared to the quantum receiver on the actual engineering design
for reference.

Theoretical Study of multi-symbol modulation receiver 2. The two yuan zero signal conditions.
Compared to a single symbol detection scheme using joint detection receiver Thought zero conditions
More research value. In this article, we will program the receiver's condition zero
Applied to the multi-pulse pulse position modulation (MPPM), OOK modulation encoded,
Coded BPSK modulation signal. On this basis, through the receiver to receive policy optimization,
We will reduce the performance of the receiver to the classic detection scheme or less.

3. The quantum receiver built experiment platform.

\enkeywords{quantum receiver, standard quantum limit, QAM modulation, quantum code, optical communication}
\end{enabstract}

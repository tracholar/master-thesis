\begin{abstract}
在经典光通信系统中,由于受到散粒噪声的影响,
通信系统接收机性能存在着经典极限——标准量子极限(standard quantum limit, SQL)。
近年来,随着量子技术的发展,利用光的量子特性设计新型的光接收机方案逐渐引起学术界的关注。
这是因为,采用量子探测和测量的技术可以获取之前使用经典探测所不能获得的信息,
这种方案可以让系统的性能突破标准量子极限,
从而获得经典检测方案所不能达到的更低的误码率和更高的信道容量。
当前对量子接收机的研究进展还比较缓慢,尚停留在理论研究和实验演示验证阶段,
仍然有许多值得研究的理论课题和试验需要研究。
\par

在本论文中,我们首先回顾一下自上个世纪六十年代以来的研究进展,
然后就其中的几个问题进行深入探究。本文主要分为以下三个部分。

\par

1. 





\keywords{量子接收机\zhspace 标准量子极限\zhspace QAM调制\zhspace 
量子编码\zhspace 光通信}

\end{abstract}

\begin{enabstract}
This is a sample document of USTC thesis \LaTeX template for bachelor, master
and doctor. The template is created by zepinglee and seisman, which orignate from
the template created by ywg@USTC. The template meets the equirements of USTC
theiss writing standards.

This document will show the usage of basic commands provided by \LaTeX and some
features provided by the template. For more information, please refer to the
template document ustcthesis.pdf.

\enkeywords{University of Science and Technology of China (USTC), Thesis, Universal \LaTeX{} Template, Bachelor, Master, PhD}
\end{enabstract}

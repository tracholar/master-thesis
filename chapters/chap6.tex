\chapter{总结与展望}
\section{研究成果总结}
在本论文研究内容的第一部分中,我们首先通过理论分析,
我们得到了QAM信号的标准量子极限和Helstrom极限的渐近性能分别为
$e^{-\alpha^2}$和$e^{-4\alpha^2}$,这里$\alpha$为相空间中,
符号间最小距离的一半,具体定义见第\ref{chp:qam}章。
由此可见如果采用最优量子检测方案,
接收机的性能将有指数倍的提升。
接着我们对Bondurant接收机、自适应分区检测接收机和混合接收机
等三类QAM信号量子接收机进行理论分析和数值仿真分析
我们发现这三类接收机都能突破标准量子极限,
且在较大信号时的理论渐近性能分析显示,这三种接收机比经典检测接收机的标准量子极限都有指数倍的收益。
相比较而言,自适应分区检测接收机最适合工程实现,
因为它只需要有限带宽的反馈控制即可,并且可以采用光子数可分辨的探测器减少分区数目并改善系统的鲁棒性。
QAM调制信号被广泛应用到大容量光通信中,
如果进一步采用量子接收机作为接收端,这将有利于进一步提升通信的信道容量,
进一步延长通信中继的距离。

在本论文研究工作的第二部分中,我们将目光从单个符号的检测放到多个符号的联合检测上来。
我们分析了OOK调制和BPSK调制下的多个符号联合检测问题,
这其中包括MPPM信号、编码后的OOK调制信号和编码后的BPSK调制信号。
首先我们通过理论分析,得出了这些信号的标准量子极限渐近性能为$e^{-\lceil d_{\min}/2\rceil \alpha^2}$,
而Helstrom极限大信号近似为$e^{-d_{\min} \alpha^2}$,这里$\alpha$为每个脉冲对应的复振幅,$d_{\min}$为最小码间距离,
详细定义见第\ref{chp:cpn}章。
接着采用Schmidt正交化的方法解决了他们的Helstrom极限数值求解问题。
最后,我们研究了针对这些调制信号的条件归零接收机的性能。
我们通过动态规划算法优化接收机的控制策略,
最终将接收机的误码率降低到经典检测方案的标准量子极限以下。
这种采用最优控制的条件归零接收机有望被应用到深空光通信中,
用来接收MPPM信号。这将极大地提高深空光通信的频谱利用率,从而进一步提高通信的信道容量。
针对二进制编码信号,这种联合监测方案将进一步降低误符号率,
能够进一步提高系统的能量效率。

在本论文研究工作的第三部分中,
我们设计了一个量子接收机实验方案,并从实验上初步验证了Kennedy接收机
的可行性。实验结果表明在我们的实验条件下,
Kenendy接收机可以突破相同系统效率的标准量子极限,最大增益达到1dB。
这为进一步的实验方案设计提供了实验经验的积累。
这种接收方案也有望应用到实际的自由空间深空光通信中,
将进一步降低误码率、提升通信系统的信道容量。

总的来说,本论文的创新点可以归纳如下:

1. 得到了QAM信号和编码后二进制调制信号的标准量子极限和Helstrom极限表达式。

2. 将Schmidt正交化应用到求解一般信号的Helstrom极限的问题当中,将求解算法复杂度降低。

3. 系统地研究了QAM信号三种量子接收机方案及其性能,证实了这三种方案都能突破标准量子极限,并且给出了工程应用的方案建议。

4. 系统地研究了MPPM信号和编码后二进制调制信号的条件归零接收机方案及其特性,将动态规划算法应用到接收策略优化当中,证实了使用条件归零接收机能够降低误码率,提升系统效率。


\section{研究工作展望}
在我们的研究过程中,
我们虽然解决了一些现有的问题,
但是仍然存在一些非常有价值的问题有待进一步研究。
对于QAM信号,我们可以将控制策略变为时变控制,
那么如何设计最佳的控制策略使得
接收机的误码率进一步接近Helstrom极限。
进一步,如何设计接收机对多元调制信号都能达到Helstrom极限。
对于联合检测方案,时变的最优控制策略仍然有待解决,
这种接收方案对极化编码能否逼近Holevo容量也是一个有待研究的问题。
最后,对于工程实现方面,如何在工程上实现精确的位移操作,
在本地实现高精度的光学相位和幅度跟踪是一个极大的挑战。
我们相信,未来的研究者在解决这些难题之后,
量子接收机应用到实际的光通信系统中将指日可待,
那时光通信的信道容量将得到进一步提升,
尤其是在能量受限和对能量效率要求很高的深空光通信中。



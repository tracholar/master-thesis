\chapter{总结与展望}
\section{研究成果总结}
在本论文中,我们首先对QAM信号量子接收机进行理论分析。
我们得出了QAM信号的标准量子极限和Helstrom极限的渐近性能分别为
$e^{-\alpha^2}$和$e^{-4\alpha^2}$,可见如果采用最优量子检测方案,
接收机的性能将有指数倍的提升。
我们又对QAM信号的Bondurant接收机、自适应分区检测接收机和混合接收机进行
理论分析和数值仿真验证,发现这三种接收机都能突破标准量子极限,
且在大信号时的理论渐近性能显示,这三种接收机比经典检测方案都有指数倍的收益。
相比较而言,自适应分区检测接收机更适合工程实现,
因为它只需要有限带宽的反馈控制即可,并且可以采用PNRD减少分区数目。
QAM信号被广泛应用到大容量光通信中,
如果采用量子接收机进行接收,将有利于进一步提升通信的信道容量,
进一步延长通信距离。

接着我们将目光从单个符号的检测放到多个符号的联合检测上来,
我们分析了OOK调制和BPSK调制下的多个符号联合检测问题,
这其中包括MPPM信号、编码后的OOK调制信号和编码后的BPSK调制信号。
我们采用条件归零接收机的结构,通过动态规划算法优化其控制策略,
最终将接收机的性能降低到经典检测方案以下。
这种采用最优控制的条件归零接收机有望被应用到深空通信中,
用来接收MPPM信号,将提高深空通信的频谱利用率,进一步提高通信的信道容量。
针对二进制编码信号,这种联合监测方案将进一步降低误符号率,
能够进一步提高系统的能量效率,增加通信的距离。

最后,我们搭建了一个量子接收机实验平台,并从实验上验证了Kennedy接收机
的可行性,为进一步的实验方案提供实验经验积累。
这种接收方案也有望应用到实际的自由空降光通信中,
能够进一步降低误码率、提升通信系统的信道容量。

\section{研究工作展望}
在我们的研究过程中,
虽然解决了一些现有的问题,
但是仍然存在一些非常有价值的问题有待进一步研究。
对于QAM信号,如何设计最佳的控制策略使得
接收机的误码率进一步接近Helstrom极限。
进一步,如何设计接收机对多元调制都能达到Helstrom极限。
对于联合检测方案,时变的最优控制策略仍然有待解决,
这种接收方案对极化编码能否逼近Holevo容量也是一个有待研究的问题。
最后,对于工程实现,如果在工程上实现精确的位移操作,
或者说在本地精确的实现光学相位和幅度跟踪是一个极大的挑战。
我们相信,未来的研究者在解决这些难题之后,
量子接收机应用到实际的通信系统中将指日可待,
那时通信的信道容量将得到进一步提升,
尤其是在能量很低和对能量效率要求很高的深空通信中。



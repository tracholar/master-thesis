\chapter{QAM信号量子接收机}
在上一章中,我们回顾了现有的一些量子接收机实现方案,
包括二元检测和多元的PSK和PPM检测量子接收机。
但是到目前为止,专门针对QAM信号设计的量子接收机
研究还比较少。然而,QAM信号具有很高的频谱效率,
已被应用到大容量光通信系统之中\cite{winzer2012high}。
因此,研究QAM信号量子接收机是一件很有必要的事情。

\section{QAM信号接收机理论极限}
\subsection{标准量子极限}
在经典通信系统中,常用外差接收机检测QAM信号,
这种接收机的性能也叫QAM信号的标准量子极限(SQL)\cite{kato1999quantum}。
设$M$阶QAM信号由$M$个相干态构成,如图\ref{fig:QAM-signals}所示,
显示的是16-QAM和36-QAM星座图。假设$M$阶QAM信号每一个
正交幅度$X$或$P$都可以取$L$个不同的值,那么$M=L^2, L=3,4,5...$。
设这$L$元基本符号集合为$\Omega = \{-(L-1) + 2(i-1) | i=1,2,...,L\}$,
那么$M$阶QAM信号
可以表示为
\begin{equation}
\ket{\alpha_{uv}} = \ket{\alpha(u + j v)}, (u, v) \in \Omega. 
\end{equation}
这里$j=\sqrt{-1}$。例如,16-QAM信号就可以用相干态表示为
\begin{equation}
\begin{split}
\ket{\alpha_{1,1}} &= \ket{\alpha(1 + j )}, \\
\ket{\alpha_{1,3}} &= \ket{\alpha(1 + j 3)}, \\
\ket{\alpha_{1,-1}} &= \ket{\alpha(1 - j )}, \\
                &\vdots                      \\
\ket{\alpha_{-3,-3}} &= \ket{\alpha(-3 - j 3)}.
\end{split}
\end{equation}
这里取$\alpha > 0$,那么M阶QAM信号信号平均光子数为
\begin{equation}
\begin{split}
n &= |\alpha|^2 \frac{1}{M}\sum_{u \in \Omega}\sum_{v \in \Omega} (u^2+v^2)\\
  &= \frac{2}{3}(M-1) |\alpha|^2.
\end{split}
\end{equation}


\begin{figure}
\centering
  \includegraphics[width=0.8\textwidth]{figures/chap3/QAM-signals}
  \caption{QAM信号星座图}
  \label{fig:QAM-signals}
\end{figure}


下面我们来计算一下QAM信号的标准量子极限。
利用式\ref{eq:Her-receiver-output}可得理想情况下,
外差接收机的概率密度函数为\cite{kato1999quantum}
\begin{equation}
p(x_c, x_s| u, v) = \frac{1}{\pi} \exp[-(x_c - u\alpha)^2 - (x_s - v\alpha)^2]
\end{equation}
这里$x_c$和$x_s$对应于式\ref{eq:Her-receiver-output}中的$\alpha_1$和$\alpha_2$,
是接收机输出的两个正交幅度观测量。
根据贝叶斯检测理论,最优判决区域为
\begin{equation}
D_{u',v'} = \{ (x_c,x_s)| D_L(u') < x_c \le D_U(u'),  D_L(v') < x_s \le D_U(v') \}
\end{equation}
这里$D_U$和$D_L$是两个上下界函数,对给定的参数$L$,
定义为
\begin{equation}
\begin{split}
D_L(u) &= \begin{cases}    
          -\infty     & u < -(L-2)  \\
          \alpha(u-1) & otherwise
         \end{cases}\\
D_U(u) &= \begin{cases} 
          \infty     & u > L-2  \\
          \alpha(u+1) & otherwise
         \end{cases}
\end{split}
\end{equation}
即将复平面划分成如图\ref{fig:QAM-domain-split}所示的$M$个判决
区域,将输出统计量落在某个区域的结果判决为在该区域内的符号。
那么,外差接收机的平均错误概率为
\begin{equation}
\begin{split}
P_e &= 1 - \frac{1}{M} \sum_{u\in\Omega}\sum_{v\in\Omega} \iint_{D_{u,v}} p(x_c,x_s|u,v) dx_c dx_s \\
    &= 1 - \frac{1}{M}[1+(L-1) \erf(\alpha)]^2.
\end{split}
\end{equation}
因此,当平均光子数很大时,即$|\alpha|^2 \gg 1$,
利用余误差函数的Chernoff界\cite{chang2011chernoff},
可得QAM信号外差接收机的渐近性能为
\begin{equation}
\begin{split}
P_e &\approx 2(1 - \frac{1}{L}) e^{-\alpha^2}.
\label{eq:QAM-SQL-approx}
\end{split}
\end{equation}

\begin{figure}
\centering
  \includegraphics[height=5cm]{figures/chap3/QAM-domain-split}
  \caption{16-QAM信号外差接收判决区域划分示意图}
  \label{fig:QAM-domain-split}
\end{figure}

\subsection{Helstrom极限}
与PSK信号和PPM信号不同的是,QAM信号不再具有几何均匀对称性,
因此,无法通过平方根检测的方法得到QAM信号Helstrom极限的解析表达式,
需要通过数值优化的方法,求解优化问题\ref{eq:Hel-SDP}。
为此,需要将信号用密度矩阵表示出来。
由于$M$个信号互不正交,因此这$M$个信号张成$M$维Hilbert空间。
因为是纯态信号,所以每一个信号都可以用一个向量表达。
为此需要选择一组标准正交基$\ket{e_i}, i=1,2,...,M$。
一种选取的方法是用Fork态做标准正交基,也可以利用Smit正交化的方法
生成一组基向量\cite{zxd2004matrix}。
这里,为方便计我们选用Fork态$\ket{n}$作为标准正交基,在这组基向量下,
每一个信号都是一个无穷维向量,
\begin{equation}
c_n = \bra{n}\ket{\alpha} = e^{-\frac{1}{2}|\alpha|^2} \frac{\alpha^n}{\sqrt{n!}}.
\end{equation}
它满足归一化条件
\begin{equation}
\sum_{n=0}^{\infty} |c_n|^2 = 1.
\end{equation}
为了便于计算,需要将这个无穷维向量截断成有限维向量。
这里采用如下准则,给定一个足够小的$\epsilon$,选取截断后向量长度$l$满足
\begin{equation}
\sum_{n=0}^{l-1} c_n \ge 1 - \epsilon.
\label{eq:epsilon-criterion}
\end{equation}
例如,当$\alpha=1, \epsilon=10^{-4}$时,信号$\ket{\alpha}$可以近似表达
为一个7维向量,
\begin{equation}
\ket{\alpha} = [0.6065, \\
    0.6065, \\
    0.4289, \\
    0.2476, \\
    0.1238, \\
    0.0554, \\
    0.0226]^T.
\end{equation}
对于$M$个QAM信号集合,在计算的时候需要将每一个
信号向量$\bm{c}_i$截断成维数相同的向量$\tilde{\bm{c}}_i$。
截断后的向量长度$L$满足
\begin{equation}
\begin{split}
L = \quad  & \max_i l_i  \\
s.t. \quad &\sum_{n=0}^{l_i-1} c_{in} \ge 1 - \epsilon, i=1,2,...,M.
\end{split}
\end{equation}
这里$c_{in}$代表第$i$个信号向量$\bm{c}_i$的第$n$维。
上式选取的长度$L$可以保证$M$个信号都满足式\ref{eq:epsilon-criterion}。

将信号表达为$L$维向量$\tilde{\bm{c}}$后,信号的密度矩阵可以表达为
\begin{equation}
\hat{\rho}_i = \tilde{\bm{c}} \tilde{\bm{c}}^T.
\end{equation}
接下来,我们就可以利用CVX工具箱\cite{cvx,gb08}求解半正定规划问题\ref{eq:Hel-SDP},
从而得到Helstrom极限对应的平均错误概率。

为了得到一些解析结论,我们采用平方根检测来近似QAM信号的最优检测。
$M$阶QAM信号的Gram矩阵由下式给出
\begin{equation}
\begin{split}
\hat{G}_{ij} &= \bra{\alpha_i}\ket{\alpha_j} \\
 &= \exp\{-\frac{1}{2} |\alpha|^2 [(u_i-u_j)^2 + (v_i-v_j)^2]\}, (i,j)=1,2,...,M.
\end{split}
\end{equation}
易知当$|\alpha|\gg1$时,除了对角元素为1之外,其他元素都很小,
我们将Gram矩阵分解为单位阵和一个$\hat{Z}$阵
\begin{equation}
\hat{G} = \hat{I} + \hat{Z}.
\end{equation}
且$\parallel \hat{Z}  \parallel \ll 1$,那么
利用矩阵幂级数可以得到Gram矩阵的平方根近似为
\begin{equation}
\hat{G}^{1/2} \approx \hat{I} + \frac{1}{2} \hat{Z} - \frac{1}{8} \hat{Z}^2.
\end{equation}
因为$\hat{Z}$的对角元素为0,所以
\begin{equation}
\hat{G}^{1/2}_{ii} \approx 1 - \frac{1}{8} \hat{Z}^2_{ii}.
\end{equation}

为了计算$\hat{Z}^2$,我们将信号按照复平面的位置分为角点、边界点、内点三类,如图\ref{fig:QAM-nearbor}(\textit{a})所示。
易知对于$M$阶QAM信号,三种点对应的数目分别为4,$4(L-2)$,$(L-2)^2$,其中$L^2=M$。
并且定义一个信号的最近邻点为与该信号在复平面内距离最近的点,
易知角点有2个最近邻点,边界点有3个最近邻点,而内点有4个最近邻点如图\ref{fig:QAM-nearbor}(\textit{b})所示。


\begin{figure}
\centering
  \includegraphics[width=0.8\textwidth]{figures/chap3/QAM-nearbor}
  \caption{QAM信号三类点和最近邻点示意图}
  \label{fig:QAM-nearbor}
\end{figure}

因为
\begin{equation}
\begin{split}
(\hat{Z}^2)_{ii} &= \sum_k Z_{ik}Z_{ki}    \\
                 &=\sum_k \exp\{- |\alpha|^2 [(u_i-u_k)^2 + (v_i-v_k)^2]\}.
\end{split}
\end{equation}
如果忽略高阶小量,那么只有最近邻点的值保留下来,所以
\begin{equation}
(\hat{Z}^2)_{ii} \approx \begin{cases}
                            2 e^{- 4|\alpha|^2} & \text{角点} \\
                            3 e^{- 4|\alpha|^2} & \text{边界点} \\
                            4 e^{- 4|\alpha|^2} & \text{内点} .
                         \end{cases}
\end{equation}
所以由\ref{eq:SRM-Pe}式,QAM信号的平方根检测渐近错误概率为
\begin{equation}
\begin{split}
P_e & \approx 1 - \frac{1}{M}\sum_{i=1}^M |1 - \frac{1}{8} (\hat{Z}^2)_{ii}|^2 \\
   & \approx 1 - \frac{1}{M}\sum_{i=1}^M (1 - \frac{1}{4} (\hat{Z}^2)_{ii}) \\
   & \approx  \frac{1}{4M} [4\times 2e^{- 4|\alpha|^2} + 4(L-2) \times 3e^{- 4|\alpha|^2} \\
   & \qquad  + (L-2)^2 \times 4e^{- 4|\alpha|^2}] \\
   & =      (1 - \frac{1}{L})  e^{- 4|\alpha|^2}. 
\label{eq:QAM-SRM-approx}   
\end{split}
\end{equation}
因为平方根检测在平均光子数较大的时候,与最优检测几乎一致,所以可以认为
该渐近性能就是Helstrom极限在平均光子数较大时的渐近性能。
比较\ref{eq:QAM-SQL-approx}式和\ref{eq:QAM-SRM-approx}式可知,
采用量子检测的手段,接收机的性能将有指数倍的增益。

\section{QAM信号Bondurant接收机}
为了探测QAM信号,我们借鉴了QPSK信号第一种Bondurant接收机的思想,
将这种接收策略应用到QAM信号的接收。
如图\ref{fig:QAM-Bondurant-receiver}所示,这种接收机由一个高透过率的波束分束器、
一个单光子探测器和一个与信号场高相干的本振构成。

\begin{figure}
\centering
  \includegraphics[width=\textwidth]{figures/chap3/Bondurant-receiver}
  \caption{QAM信号Bondurant接收机和反馈策略示意图}
  \label{fig:QAM-Bondurant-receiver}
\end{figure}

为了说明接收机接收过程,我们将$M$阶QAM信号按顺序编号为$\alpha_1,\alpha_2,...,\alpha_M$。
记符号到来的初始时刻为$t=0$,符号持续的时间为$T$。
接收机按照图\ref{fig:QAM-Bondurant-receiver}所示的策略
进行反馈控制,反馈策略可以归纳如下:

1.在$t=0$时刻,控制本振使得位移操作$\hat{D}(\beta) = \hat{D}(-\alpha_1)$,
  即将符号1归零,同时选择假设$H_1$:当前的符号为$\alpha_1$。
  
2.在任意$0<t<T$时刻,如果没有发生光子计数事件,
  那么保持当前的假设和本振,继续接收信号;
  如果在$t_i$时刻发生了第$i$个光子计数事件,
  那么将当前的假设由$H_i$转变为$H_{i+1}$,
  同时控制本振归零下一个符号,即$\hat{D}(\beta) = \hat{D}(-\alpha_{i+1})$。
  然后继续接收信号。
  
3.重复步奏2直到$t=T$时刻,信号接收完毕,
  输出当前的假设$H_i$作为对该符号的判决。
  
假设发送的信号是$\alpha_k$,
在该符号周期内,发生了$k'$个光子计数事件,
这$k'$个光子计数事件发生的时间为$t_i,(i=1,2,...,k')$,
仅当$k' = k-1$的时候,接收机判决输出的假设为$H_k$,
它对应的概率分布为
\begin{equation}
p(t_1,t_2,...,t_{k-1}|k) = \prod_{i=1}^{k-1} w(t_i|k).
\end{equation}
假定$t_0=0$,这里
\begin{equation}
w(t_i|k) =  \frac{|\alpha_k - \alpha_i|^2}{T} \exp(-|\alpha_k - \alpha_i|^2 \frac{t_i-t_{i-1}}{T}) .
\label{eq:w-cond}
\end{equation}
代表在发送的信号为$\alpha_k$时,
在半开时间区间$(t_{i-1}, t_i]$内,
恰好在$t_i$时刻发生了光子计数
的概率密度。
所以,在发送的信号为$\alpha_k$时接收机判决为$\alpha_k$的
条件概率为
\begin{equation}
\Pr{k|k} = \int\int\cdots\int_{0<t_1<t_2<\cdots<t_{k-1}<T} p(t_1,t_2,...,t_{k-1}|k) dt_1 dt_2 \cdots dt_{k-1}.
\label{eq:QAM-cond-prob}
\end{equation}

对于第$k$个信号,上述积分为$k-1$重,
为了便于分析,对给定的符号$\ket{\alpha_k}$定义如下函数
\begin{equation}
Q_i^k(t) = \begin{cases}
            1 & i=1, \\
            \int_t^1  dt_{i-1}' n_{k,i-1} e^{-n_{k,i-1}(1-t)} Q_{i-1}(t_{i-1}') & 1 < i \e k.
         \end{cases}
\label{eq:Q-func}
\end{equation}
上式中$n_{k,l} = |\alpha_k - \alpha_l|^2$。
接下来,我们利用\ref{eq:w-cond}式将\ref{eq:QAM-cond-prob}式进行改写,
作变量代换$t_i \rightarrow T t_{k-i}'$,那么\ref{eq:QAM-cond-prob}式化为
\begin{equation}
\begin{split}
\Pr{k|k} = & \int_{0}^1 dt_{k-1}' n_{k,1} e^{-n_{k,1} t_{k-1}'}          \int_{t_{k-1}'}^1 dt_{k-2}'   \frac{n_{k,2}}{T} e^{-n_{k,2} (t_{k-2}'-t_{k-1}')} \\
           & \cdots    \int_{t_1'}^1 dt_1' n_{k,k-1} e^{-n_{k,k-1} (t_1'-t_2')}.
\end{split}
\end{equation}

对式\ref{eq:Q-func}反复递归可得$Q_k(t)$表达式,对照上式有$\Pr{k|k} = Q_k(0)$。
因此,我们可以用递归式\ref{eq:Q-func}来求上述积分。

下面,我们来证明一个断言
\begin{equation}
\begin{split}
Q_m^k(t) = 1 + \sum_{i=1}^{m-1} A_i e^{-n_{k,i}(1-t)}.
\end{split}
\end{equation}


为了精确求解接收机的平均错误概率,需要计算多重积分\ref{eq:QAM-cond-prob},
对于低维情况如16-QAM信号,还能在比较快的时间里计算出来,
但是对于更高维情况如36-QAM、64-QAM信号,
计算时间较慢。这里采用Monte Carlo仿真的方法计算接收机的性能,
这种方法的计算精度与问题的维度无关\cite{mcbook},
计算时间只随仿真点数和信号数目线性增长,
能够有效求解高阶QAM接收机平均错误概率的问题。
每一次仿真都是一个独立重复事件,用随机变量$\xi$
表示输出结果,$\xi=0$表示接收机判断错误,
$\xi=1$表示接收机判断正确。那么有
\begin{equation}
\E \xi = p, \Var \xi = p(1-p).
\end{equation}
重复该独立事件$N$次,得到估计值
\begin{equation}
\tilde{p} = \frac{1}{N} \sum_{i=1}^N \xi_i.
\end{equation}
该统计量的期望与方差为
\begin{equation}
\E \tilde{p} = p, \Var \tilde{p} = \frac{p(1-p)}{N}.
\end{equation}
若$p \ll 1$,那么它的标准差约为$\sqrt{p/N}$,
为了得到一定精度的仿真结果,可以选取适当的$N$
和一个小量$\epsilon$,使得$\Delta p / p < \epsilon$。
当$N$很大时,随机变量$\tilde{p}$可以认为近似服从正态分布,
可以利用$3\sigma$原则选取$\Delta p = 3 \sqrt{\Var \tilde{p}}$
因此若取$\epsilon=0.1$计算精度,那么有
\begin{equation}
N > \frac{300}{p}.
\end{equation}
例如,要在平均错误概率大约在$10^{-5}$的时候,
保证$\epsilon=0.1$的计算精度,需要的仿真点个数为$3\times10^7$。

我们以16-QAM信号和36-QAM信号位移,
通过数值仿真分析,得到Bondurant接收机的性能曲线如图\ref{fig:QAM-bondurant-error}
所示。从图中可以看到,当平均光子数较小的时候,
Bondurant接收机性能没有突破标准量子极限,
当平均光子数较大时,Bondurant接收机的性能
就可以突破标准量子极限,并且随着光子数的增大,
这种优势就越大。

\begin{figure}
\centering
  \includegraphics[width=\textwidth]{figures/chap3/QAM-bondurant-error}
  \caption{QAM信号Bondurant接收机性能曲线}
  \label{fig:QAM-bondurant-error}
\end{figure}


\section{QAM信号自适应分区检测接收机}


\section{QAM信号混合接收机}


\section{三种接收机的对比}


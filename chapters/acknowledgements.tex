\begin{acknowledgements}
光阴似箭,我的七年学习生涯转眼间就快结束了。
遥想当年刚进入科大时的意气风发,对未来的
学习生活充满憧憬。在这毕业之际,一时感慨万千。
首先,我要衷心感谢我的导师朱冰教授。自从
本科进入实验室开始,在这将近4年的时间里,
朱冰老师对我学习上和科研上给予了悉心的指导,
帮助我从科研的大门口开始了真正意义上的研究工作。
朱老师治学严谨、科研经历丰富,善于发掘本质问题,
常常一针见血指出科研中的问题。本论文的完成,从选题到
研究工作的完成过程,都离不开朱老师的细心指导。

感谢实验室杨利老师,她学术造诣深厚,为人亲切,
在研究生期间给予我亲切的鼓励。感谢实验室苏觉老师,
他对我的鼓励让我感受到实验室的人情味。

感谢实验室已毕业的李科师兄,在大四、研一和研二
期间,亲自带领我从量子接收机的入门课题开始,
到后续的深入研究。在整个过程中,
李科师兄总是不厌其烦地解答我研究中的问题。
即使在毕业之后,也不断的督促我和指导我的研究工作。
另外,我也要感谢已毕业的王晓飞师兄、许花醒师兄,
在实验室的时候给我的指导和帮助。

感谢研究生师兄陈林勋、朱圣强,与在实验室
与他们一起科研和生活的日子让我记忆深刻。
他们对学习的认真对我影响深刻,在找工作期间
也作为过来人给予我细心的指导。

感谢博士师姐陈田,她科研的认真和永不放弃的毅力
给我留下深刻的印象。在科研和论文完成期间也
和她多番讨论,这些讨论给与我灵感和帮助。
感谢惠君和仇南师弟,在实验室期间一直给与我实验上的支持。

感谢实验室的师弟师妹们,与他们在一起的研究生生活,
让我感受到家的温暖。

感谢远在纽约的大学同学王洋,在我完成会议论文期间,
给予我英语上的帮助,感谢侯俊峰、文士学、朱戎生等同学,
和他们一起学习,度过这七年的校园生活,让我倍感温馨。

感谢我的父母,没有他们的支持,我是不可能顺利完成学业的,
感谢我的女友对我学习的支持和鼓励。


\end{acknowledgements}
